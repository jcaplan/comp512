%%%%%%%%%%%%%%%%%%%%%%%%%%%%%%%%%%%%%%%%%
% Beamer Presentation
% LaTeX Template
% Version 1.0 (10/11/12)
%
% This template has been downloaded from:
% http://www.LaTeXTemplates.com
%
% License:
% CC BY-NC-SA 3.0 (http://creativecommons.org/licenses/by-nc-sa/3.0/)
%
%%%%%%%%%%%%%%%%%%%%%%%%%%%%%%%%%%%%%%%%%

%----------------------------------------------------------------------------------------
%	PACKAGES AND THEMES
%----------------------------------------------------------------------------------------

\documentclass[10pt]{beamer}

\mode<presentation> { 

% The Beamer class comes with a number of default slide themes
% which change the colors and layouts of slides. Below this is a list
% of all the themes, uncomment each in turn to see what they look like.

%\usetheme{default}
%\usetheme{AnnArbor}
%\usetheme{Antibes}
%\usetheme{Bergen}
% \usetheme{Berkeley}
%\usetheme{Berlin}
%\usetheme{Boadilla}
%\usetheme{CambridgeUS}
% \usetheme{Copenhagen}
%\usetheme{Darmstadt}
%\usetheme{Dresden}
%\usetheme{Frankfurt}
%\usetheme{Goettingen}
%\usetheme{Hannover}
%\usetheme{Ilmenau}
%\usetheme{JuanLesPins}
%\usetheme{Luebeck}
% \usetheme{Madrid}
%\usetheme{Malmoe}
%\usetheme{Marburg}
\usetheme{Montpellier}
%\usetheme{PaloAlto}
%\usetheme{Pittsburgh}
%\usetheme{Rochester}
%\usetheme{Singapore}
%\usetheme{Szeged}
%\usetheme{Warsaw}

% As well as themes, the Beamer class has a number of color themes
% for any slide theme. Uncomment each of these in turn to see how it
% changes the colors of your current slide theme.

% \usecolortheme{albatross}
% \usecolortheme{beaver}
% \usecolortheme{beetle}
% \usecolortheme{crane}
% \usecolortheme{dolphin}
% \usecolortheme{dove}
% \usecolortheme{fly}
% \usecolortheme{lily}
% \usecolortheme{orchid}
% \usecolortheme{rose}
% \usecolortheme{seagull}
% \usecolortheme{seahorse}
% \usecolortheme{whale}
% \usecolortheme{wolverine}

%\setbeamertemplate{footline} % To remove the footer line in all slides uncomment this line
%\setbeamertemplate{footline}[page number] % To replace the footer line in all slides with a simple slide count uncomment this line

%\setbeamertemplate{navigation symbols}{} % To remove the navigation symbols from the bottom of all slides uncomment this line
}

\usepackage[]{algorithm2e}

\usepackage{graphicx} % Allows including images
\usepackage{booktabs} % Allows the use of \toprule, \midrule and \bottomrule in tables
% \usepackage{subcaption}
% \usepackage{float}
\usepackage[font=scriptsize]{subfig}
\usepackage[font=scriptsize,labelfont=bf]{caption}

\usepackage{courier} % Required for the courier font
\usepackage{listings}


\definecolor{mygreen}{rgb}{0,0.6,0}
\definecolor{mygray}{rgb}{0.5,0.5,0.5}
\definecolor{mymauve}{rgb}{0.58,0,0.82}
\definecolor{mylisting}{rgb}{1,0.98,0.756}
\lstset{ %
  backgroundcolor=\color{mylisting},   % choose the background color; you must add \usepackage{color} or \usepackage{xcolor}
  basicstyle=\footnotesize\ttfamily,        % the size of the fonts that are used for the code
  breakatwhitespace=false,         % sets if automatic breaks should only happen at whitespace
  breaklines=true,                 % sets automatic line breaking
  captionpos=b,                    % sets the caption-position to bottom
  commentstyle=\color{mygreen},    % comment style
  deletekeywords={...},            % if you want to delete keywords from the given language
  escapeinside={\%*}{*)},          % if you want to add LaTeX within your code
  extendedchars=true,              % lets you use non-ASCII characters; for 8-bits encodings only, does not work with UTF-8
  frame=single,                    % adds a frame around the code
  keepspaces=true,                 % keeps spaces in text, useful for keeping indentation of code (possibly needs columns=flexible)
  keywordstyle=\color{blue},       % keyword style
  language=Java,                 % the language of the code
  morekeywords={*,...},            % if you want to add more keywords to the set
  numbers=left,                    % where to put the line-numbers; possible values are (none, left, right)
  numbersep=5pt,                   % how far the line-numbers are from the code
  numberstyle=\tiny\color{mygray}, % the style that is used for the line-numbers
  rulecolor=\color{black},         % if not set, the frame-color may be changed on line-breaks within not-black text (e.g. comments (green here))
  showspaces=false,                % show spaces everywhere adding particular underscores; it overrides 'showstringspaces'
  showstringspaces=false,          % underline spaces within strings only
  showtabs=false,                  % show tabs within strings adding particular underscores
  stepnumber=2,                    % the step between two line-numbers. If it's 1, each line will be numbered
  stringstyle=\color{mymauve},     % string literal style
  tabsize=2,                       % sets default tabsize to 2 spaces
  title=\lstname ,                  % show the filename of files included with \lstinputlisting; also try caption instead of title
  moredelim=**[is][\color{red}]{@!}{@!},
}



\newcommand\toc{\begin{frame}
\tableofcontents[
  currentsection,
  sectionstyle=show/shaded,
  subsectionstyle=show/shaded/shaded
]
\end{frame}
}

\setbeamertemplate{headline}{}
\setbeamertemplate{navigation symbols}{}%remove navigation symbols
\setbeamertemplate{footline}[frame number]
\setbeamertemplate{bibliography item}{[\theenumiv]}


\newcommand{\addfigure}[2]{

\begin{figure}[h]
	\centering
	\includegraphics[scale=#1]{figures/#2} 
\end{figure}

}



%----------------------------------------------------------------------------------------
%	TITLE PAGE
%----------------------------------------------------------------------------------------

\title[Comp 512 Project - Milestone 2]{Comp 512 Project - Milestone 3} % The
% short title appears at the bottom of every slide, the full title is only on the title page

\author{Jonah Caplan, HaiTong Yang - Group 28} % Your name
\institute % Your institution as it will appear on the bottom of every slide, may be shorthand to save space
{
McGill University \\ % Your institution for the title page
\medskip
\textit{jonah.caplan@mail.mcgill.ca,hai.yang@mail.mcgill.ca} % Your email address
}
\date{December 3, 2015} % Date, can be changed to a custom date

\begin{document}

\begin{frame}
\titlepage % Print the title page as the first slide
\end{frame}

\begin{frame}
\frametitle{System Overview}
\addfigure{0.5}{arch.pdf}
\end{frame}

\begin{frame}
\frametitle{System Overview}
\addfigure{0.5}{system.pdf}
\end{frame}

\begin{frame}
\frametitle{2PC}
A successful read and write operation.
\addfigure{0.3}{sequence.pdf}
\end{frame}


\begin{frame}
\frametitle{Timeout algorithm at MW}
\SetKwFor{Upon}{upon}{do}{end}
\begin{algorithm}[H]
%  \KwData{this text}
%  \KwResult{how to write algorithm with \LaTeX2e }
 \Upon{initialization}{
 	Timer t\;
 	List of TimerTask tList\;
 } 
 \Upon{Start(txnID)}{
  add new TimerTask for txnID to tList and schedule task\;
  }
  \Upon{enlist(txnId)}{
  cancel previous task\;
  schedule new task\;
  }
  \Upon{commit(txnId) or abort(txnId)}{
  cancel previous task\;
  remove txnId from list\;
  }
  \Upon{timerExpires(txnId)}{
  	abort(txnId)\;
  }
 
 
 \end{algorithm}

(basically the same for RM)
\end{frame}

\begin{frame}
\frametitle{Implementing and Testing Crash Scenarios}
\begin{itemize}
  \item Starting and stopping tomcat servers is expensive.
  \item Don't know how to kill/restart app instead of server.
  \item Want to debug across network interfaces.
  \item Implement testbench that bypasses web services.
  \item Comprehensive debugging, JUnit, etc. now possible.
\end{itemize}
\addfigure{0.3}{crash-class.pdf}
\end{frame}

\begin{frame}[fragile]
\frametitle{Running crash scenarios}
\begin{itemize}
  \item Bypass middleware and run tests through Transaction Manager and
  MWTestClient.
  \item Assign unique ID to each point in program where crash should be
  injected.
  \item Client (or testbench) specifies crash location at beginning of scenario.
\end{itemize}

\begin{lstlisting}
id = tm.start();
carPrice = 102;
numCars = 37;
carClient.addCars(id, location, numCars, carPrice);
int roomPrice = 103;
int numRooms = 99;
roomClient.addRooms(id, location, numRooms, roomPrice);
@!tm.setCrash(new TestTMCrash());
tm.setCrashLocation(crashLocation);@!
try{
  result = tm.commit(id);
} catch (CrashException e){
  TMClient.deleteInstance();
  LockManager.reset();
  System.out.println("commit crashed!");
}
\end{lstlisting}

\end{frame}

\begin{frame}
\frametitle{Handling crashes}
\textbf{Crashes in RM:}
\begin{itemize}
  \item \texttt{prepareVote()} fails on any RM $\rightarrow$ TM aborts
  \item \texttt{abort()} or \texttt{commit()} fails on any RM $\rightarrow$
  TM repeats requests.
\end{itemize}

\textbf{Crashes in TM:}
\begin{itemize}
  \item if \texttt{prepareVote()} never excecuted $\rightarrow$ RM txn will
  timeout.
  \item if \texttt{prepareVote()} executed $\rightarrow$ RM txn waits
  indefinitely.
\end{itemize}

\textbf{Limitations:}
\begin{itemize}
  \item Web services made it difficult to implement messages from RM to MW
  (limited granularity for crash injection).
  \item Assume logging never fails.
\end{itemize}
\textbf{Improvements:}
\begin{itemize}
  \item TM should take note in case where RM does not reply after voting yes
  successfully and move on as if RM voted yes, while periodically checking RM
  status.
\end{itemize}

\end{frame}


\begin{frame}
\frametitle{RM Recovery}
\textbf{RM strategy:}
\begin{itemize}
  \item Read record when starting up.
  \item Restore any transaction that voted yes to \texttt{prepareCommit()}.
  \item Abort any transaction that started but did not vote.
\end{itemize}

\textbf{TM strategy:}
\begin{itemize}
  \item If TM sends vote after reboot, RM will find abort record and vote no.
  \item If TM sends commit or abort after reboot, then appropriate action is
  taken.
\end{itemize}

\textbf{Storing update for txn $t$}
\begin{enumerate}
  \item Check current shadow copy $S$.
  \item Copy $S' \leftarrow S$.
  \item Use write List for $t$ to update $S'$.
  \item Write $S'$ to disk.
\end{enumerate}

\end{frame}


\begin{frame}
\frametitle{Room for improvement}

\begin{itemize}
  \item Main copy + shadow copy must fit in memory at same
time. 
\item Better to find lower level solutions (e.g. DMA copy and file indexing to
allow in place modification of shadow copy).
\item Keep record for changes to shadow copy to allow targeted modifications
instead of blindly copying everything. 
\end{itemize}

\end{frame}

% \begin{frame}
% \frametitle{Performance results}
% 
% \addfigure{0.4}{perf-txn.pdf}
% \end{frame}
% 
% \begin{frame}
% \frametitle{Where number of clients becomes the problem}
% \addfigure{0.4}{perf-client.pdf}
% \end{frame}

\end{document} 
